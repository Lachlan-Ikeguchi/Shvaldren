\documentclass{book}
\usepackage[a4paper, margin=2cm]{geometry}
\usepackage[british]{babel}
\usepackage{parskip}
\usepackage{graphicx}
\usepackage{hyperref}

\graphicspath{ {./letters/} }

\title{Shvaldren}
\author{Lachlan Ikeguchi}

\newcommand{\letter}[1]{\begin{center}\resizebox{1cm}{!}{\includegraphics{#1}}\end{center}}

\begin{document}
\maketitle
\tableofcontents

%%%
\chapter{Overview}
There are 24 letters in the Shvaldren language.  The overall idea of this conlang is to be as flowing and to be as phonetically aesthetic as possible.  The script of the language is designed for cursive writing, and therefore omits accents and letters which cannot be written continuously.

%%%

\chapter{Vowels}
\section{Letter 1}
Close front unrounded vowel

\letter{letters/vowels/1.JPG}


\section{Letter 2}
Near-close near-front unrounded vowel

\letter{letters/vowels/2.JPG}


\section{Letter 3}
Close-mid front unrounded vowel

\letter{letters/vowels/3.JPG}


\section{Letter 4}
Open-mid front unrounded vowel

\letter{letters/vowels/4.JPG}


\section{Letter 5}
Open front unrounded vowel

\letter{letters/vowels/5.JPG}


\section{Letter 6}
Mid-central vowel

\letter{letters/vowels/6.JPG}


\section{Letter 7}
Close back rounded vowel

\letter{letters/vowels/7.JPG}


\section{Letter 8}
Close-mid back rounded vowel

\letter{letters/vowels/8.JPG}


\section{Letter 9}
Near-close near-back vowel

\letter{letters/vowels/9.JPG}


\section{Letter 10}
Open-mid back rounded vowel

\letter{letters/vowels/10.JPG}


%%%

\chapter{Consonants}
\section{Letter 11}
Bilabial nasal

\letter{letters/consonants/11.JPG}


\section{Letter 12}
Alveolar nasal

\letter{letters/consonants/12.JPG}


\section{Letter 13}
Voiceless alveolar sibilant

\letter{letters/consonants/13.JPG}


\section{Letter 14}
Voiceless retroflex sibilant

\letter{letters/consonants/14.JPG}


\section{Letter 15}
Voiceless palato-alveolar sibilant

\letter{letters/consonants/15.JPG}


\section{Letter 16}
Alveolar tap

\letter{letters/consonants/16.JPG}


\section{Letter 17}
Alveolar trill

\letter{letters/consonants/17.JPG}


\section{Letter 18}
Uvular trill

\letter{letters/consonants/18.JPG}


\section{Letter 19}
Palatal approximant

\letter{letters/consonants/19.JPG}


\section{Letter 20}
Voiced labio-velar approximant

\letter{letters/consonants/20.JPG}


\section{Letter 21}
Aveolar lateral approximant

\letter{letters/consonants/21.JPG}


\section{Letter 22}
Aveolar approximant

\letter{letters/consonants/22.JPG}


\section{Letter 23}
Voiceless glottal fricative

\letter{letters/consonants/23.JPG}


\section{Letter 24}
Voiceless alveolar plosive

\letter{letters/consonants/24.JPG}


%%%

\chapter{Grammer}
\section{Punctuation}


\section{Structure of a word}


\section{Structure of a sentence}


\end{document}

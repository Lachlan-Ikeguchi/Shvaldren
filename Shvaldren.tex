\documentclass{book}
\usepackage[a4paper, margin=2cm]{geometry}
\usepackage[british]{babel}
\usepackage{parskip}
\usepackage{graphicx}
\usepackage{hyperref}

\title{Shvaldren}
\author{Lachlan Ikeguchi}

\newcommand{\display}[1]{\begin{center}\resizebox{2cm}{!}{\includegraphics{#1}}\end{center}}

\begin{document}
\maketitle
\tableofcontents

%%%
\chapter{Overview}
There are 24 letters in the Shvaldren language.  The overall idea of this conlang is to be as flowing and to be as phonetically aesthetic as possible.  The script of the language is designed for cursive writing, and therefore omits accents and letters which cannot be written continuously.

%%%

\chapter{Vowels}
\section{Letter 1}
Close front unrounded vowel

\display{letters/vowels/1.JPG}


\section{Letter 2}
Near-close near-front unrounded vowel

\display{letters/vowels/2.JPG}


\section{Letter 3}
Close-mid front unrounded vowel

\display{letters/vowels/3.JPG}


\section{Letter 4}
Open-mid front unrounded vowel

\display{letters/vowels/4.JPG}


\section{Letter 5}
Open front unrounded vowel

\display{letters/vowels/5.JPG}


\section{Letter 6}
Mid-central vowel

\display{letters/vowels/6.JPG}


\section{Letter 7}
Close back rounded vowel

\display{letters/vowels/7.JPG}


\section{Letter 8}
Close-mid back rounded vowel

\display{letters/vowels/8.JPG}


\section{Letter 9}
Near-close near-back vowel

\display{letters/vowels/9.JPG}


\section{Letter 10}
Open-mid back rounded vowel

\display{letters/vowels/10.JPG}


%%%

\chapter{Consonants}
\section{Letter 11}
Bilabial nasal

\display{letters/consonants/11.JPG}


\section{Letter 12}
Alveolar nasal

\display{letters/consonants/12.JPG}


\section{Letter 13}
Voiceless alveolar sibilant

\display{letters/consonants/13.JPG}


\section{Letter 14}
Voiceless retroflex sibilant

\display{letters/consonants/14.JPG}


\section{Letter 15}
Voiceless palato-alveolar sibilant

\display{letters/consonants/15.JPG}


\section{Letter 16}
Alveolar tap

\display{letters/consonants/16.JPG}


\section{Letter 17}
Alveolar trill

\display{letters/consonants/17.JPG}


\section{Letter 18}
Uvular trill

\display{letters/consonants/18.JPG}


\section{Letter 19}
Palatal approximant

\display{letters/consonants/19.JPG}


\section{Letter 20}
Voiced labio-velar approximant

\display{letters/consonants/20.JPG}


\section{Letter 21}
Aveolar lateral approximant

\display{letters/consonants/21.JPG}


\section{Letter 22}
Aveolar approximant

\display{letters/consonants/22.JPG}


\section{Letter 23}
Voiceless glottal fricative

\display{letters/consonants/23.JPG}


\section{Letter 24}
Voiceless alveolar plosive

\display{letters/consonants/24.JPG}


%%%

\chapter{Grammar}
\section{Punctuation}
\subsection{End of paragraph}
The punctuation to mark the end of a paragraph is:

\display{punctuation/paragraph-end.png}

\subsection{End of sentence}
The punctuation to mark the end of a sentence is:

\display{punctuation/sentence-end.png}

\subsection{Line break}
The punctuation to signal a break in the line is:

\display{punctuation/line-break.png}

It is used before the break and after.

\section{Structure of text}
There exists paragraphs and sentences.

\section{Structure of a sentence}
The sentences follow an Object, Verb, Subject structure.  A prefix to the verb reverses the order to the Subject, Verb, Object order.

\section{Structure of a word}
The possible composition are CVC, and CVCVC, where consonant clusters have a maximum of 2 exept in the case of the second structure where the middle consonant cluster may be up to 3 consonants.  Prefixes and suffixes follow the structure cv-, and -vc.  Compound words may be formed by the first word modified with a suffix to signify it.  The prefix and suffix system is comprehensive and is used to create most of the words.


\end{document}
